\documentclass[english,serif,mathserif,usenames,dvipsnames]{beamer}
\usetheme[informal]{gc3}

\usepackage[T1]{fontenc}
\usepackage[utf8]{inputenc}
\usepackage{babel}

%% This is optional: it adds a few commands and environment we
%% regularly use in our slide sets
\usepackage{gc3}

\begin{document}

%% Optional Argument in [Brackets]: Short Title for Footline
\title[A Wonderful Slide Set]{A Very Wonderful Slide Set for the Benefit of the Audience}
\subtitle{Or: How to Demo the GC3 Beamer Theme}
\author{A. U. Thor \texttt{<a.u.thor@uzh.ch>}}
\date{\today}

%% Makes the title slide
\maketitle

\section{Initial}
\begin{frame}{The title of this slide can take up to two lines}

  Lorem ipsum dolor sit amet, consectetur adipiscing elit.
  \begin{itemize}
  \item dummy text
  \item dummy text
  \end{itemize}

  No slide is a real slide without some math:
  $$x/2 = \int_{0}^{1} ydy$$
  and a \emph{(clickable!)} web URL: \url{www.uzh.ch}

\end{frame}

% chapter division
\section{Features}
\part{A part divider slide}

\begin{frame}[fragile,fragile]
  \frametitle{Features available in \texttt{gc3.sty}}

  \begin{python}
# blocks of Python code
def factorial(n):
  if n > 0:
    # with highlight too!
    ~\HL{return n*factorial(n-1)}~
  else:
    return 1
  \end{python}

  \+
  \HL{Highlighting} works in normal text as well, and
  in \HL[green!25]{different} \HL[yellow!25]{colors}.

  \+
  The \lstinline|\+| command inserts separates paragraphs with more
  vertical space.
\end{frame}

\begin{frame}[fragile]
  \frametitle{More features from gc3.sty}

  Environments for typesetting exercises, questions, and references to additional material.

  \begin{question}
    How do you typeset an exercise?
  \end{question}

  \+
  \begin{exercise}
    Use \lstinline|\begin{exercise}| and \lstinline|\end{exercise}| to
    typeset an exercise.
  \end{exercise}

  \begin{seealso}
    The \texttt{README.md} file in this directory.
  \end{seealso}

  \+
  To use the optional features, insert this in the \LaTeX{} preamble:
\begin{lstlisting}[language=tex]
\usepackage{gc3}
\end{lstlisting}
\end{frame}


\section{The End}
\begin{frame}[fragile]
  \frametitle{You have seen\ldots}
  \tableofcontents[sectionstyle=show/shaded]
\end{frame}

\end{document}

%%% Local Variables:
%%% mode: latex
%%% TeX-master: t
%%% End:
